\subsection{Approssimazioni lineari}

\subsubsection{Introduzione all'approssimazione tramite Taylor}

Sia \(A \subseteq \Rn \) con \(A\) aperto;

Sia \(f: A \subseteq \Rn \rightarrow \R \), di classe \(C^{k}\) per qualche \(k \in \N \);

Siano \(\ux \) e \(\ux +\uh \) due punti in \(A\) tali che il segmento \([\ux, \ux +\uh ] \subset \Rn \) sia tutto contenuto in \(A\).

Definiamo la funzione \(x(t) : \R \to \Rn \), che associa \(t \mapsto \ux+t\uh \) definita \(\forall t \in [0,1]\).

Possiamo quindi scrivere gli elementi del segmento come:
\[
    [\ux , \ux +\uh ] = \{x(t) ~\forall t \in [0,1]\} \subset \Rn
\]

Definiamo inoltre la funzione \(F(t) : [0,1] \to \R \), che associa \(t \mapsto f(x(t)) = f(\ux+t\uh)\)

Se \(f \in C^{1}\), allora anche \(F \in C^1\), quindi esiste:

\[
    F'(t) = [f(x(t))]' = \prods{\nabla f(x(t))}{\frac{\diff }{\diff t}x(t)} = \prods{\nabla f(\ux+t\uh)}{\uh} = \sum^{n}_{i=1} \left[ \frac{\partial f}{\partial x_i} (\ux + t\uh ) \right] h_i
\]

Se inoltre, \(f \in C^{2}\), allora esiste anche:

\[
    F''(t) = \left( F'(t) \right)' = \sum^{n}_{j=1} \frac{\partial}{\partial x_j} \left[ \sum^{n}_{i=1} \frac{\partial f}{\partial x_i}(\ux + t\uh ) h_i \right] h_j = \sum^{n}_{i,j=1} \left[ \frac{\partial^{2} f}{\partial x_i \partial x_j} (\ux +  t\uh ) \right] h_i h_j
\]

Questo è un polinomio quadratico.

\pagebreak
\subsubsection{Polinomio di Taylor con resto di Lagrange}

Ridefiniamo quanto detto nell'introduzione:
\begin{itemize}
    \item Sia \(A \subseteq \Rn \) con \(A\) aperto;

    \item Sia \(f: A \subseteq \Rn \rightarrow \R \), di classe \(C^{k}\) per qualche \(k \in \N \);

    \item Siano \(\ux \) e \(\ux +\uh \) due punti in \(A\) tali che il segmento \([\ux, \ux +\uh ] \subset \Rn \) sia tutto contenuto in \(A\).

    \item Definiamo la funzione \(x(t) : \R \to \Rn \), che associa \(t \mapsto \ux+t\uh \) definita \(\forall t \in [0,1]\). Con questa, il segmento è possibile scriverlo come:
          \[
              [\ux , \ux +\uh ] = \{x(t) ~\forall t \in [0,1]\} \subset \Rn
          \]

    \item Definiamo inoltre la funzione \(F(t) : [0,1] \to \R \), che associa \(t \mapsto f(x(t)) = f(\ux+t\uh)\)
\end{itemize}

Osserviamo che \(F(0) = f(\ux)\) e che \(F(1) = f(\ux+\uh)\).

\proposizione{Formula di Taylor con resto di Lagrange}{
    Essendo \(F\) una funzione in una variabile, applico Taylor in una variabile fino all'ordine \((k-1)\), usando il resto di Lagrange (rappresentato dentro il box).

    Quindi, centrando il polinomio di Taylor in \(0\), esiste \(\theta \in (0,1)\) tale che vale:
    \begin{align*}
        F(1) & = F(0) + F'(0)(1-0) + \frac{F''(0)}{2}{(1-0)}^{2} + \cdots + \frac{F^{(k-1)}(0)}{(k-1)!}{(1-0)}^{k-1} + \boxed{\frac{F^{(k)}(\theta)}{k!}{(1-0)}^{k}} \\
             & = F(0) + F'(0) + \frac{F''(0)}{2} + \cdots + \frac{F^{(k-1)}(0)}{(k-1)!} + \boxed{\frac{F^{k}(\theta)}{k!}}
    \end{align*}
}

Vediamo dunque di dimostrare quanto appena detto per i primi ordini:

\filbreak{}
\subsubsection*{Per k=1}

Con \(k=1\) ritroviamo una versione in più variabili del teorema di Lagrange.

\teorema{Formula di Taylor di ordine 1 con resto in forma di Lagrange}{
    Date tutte le definizione iniziali;
    Se \(f \in C^1\), allora esiste \(\theta \in (0,1)\) tale che:
    \begin{align*}
        f(\ux+\uh) & = f(\ux) + f_x(\theta)\cdot\uh                                                                    &                         \\
                   & = f(\ux) + \underbrace{\prods{\nabla f(\ux+ \theta \uh)}{\uh}}_{\text{resto calcolato nel punto}} & \text{(forma compatta)} \\
                   & = f(\ux) + \sum^{n}_{i=1} \frac{\partial f}{\partial x_i}(\ux+\theta\uh) \cdot h_i                & \text{(forma estesa)}
    \end{align*}
}
\begin{proof}
    Applichiamo Lagrange a \(F\) nell'intervallo \((0,1)\); \(\exists \theta \in (0,1)\) tale che:

    \[
        F'(\theta) = \frac{F(1) - F(0)}{1-0} \implies F(1) = F(0) + F'(\theta)
    \]

    Ricordando che \(F'(t) = \prods{\nabla f(\ux+t\uh)}{\uh}\), e che \(F(0) = f(\ux)\), e che \(F(1) = f(\ux+\uh)\); sostituendo, ritroviamo la tesi:

    \[
        f(\ux+\uh) = f(\ux) + \prods{\nabla f(\ux+ \theta \uh)}{\uh}
    \]

\end{proof}

\subsubsection*{Per k=2}

\teorema{Formula di Taylor di ordine 1 con resto in forma di Lagrange}{
    Date tutte le definizione iniziali;
    Se \(f \in C^2\), allora esiste \(\theta \in (0,1)\) tale che:
    \begin{align*}
        f(\ux+\uh) & = f(\ux) + \sum^{n}_{i=1} \left[ \frac{\partial f}{\partial x_i}(\ux) \cdot h_i \right] + \frac{1}{2}\sum^{n}_{i,j=1} \left[ \frac{\partial^{2} f}{\partial x_i \partial x_j}(\ux+\theta\uh) \cdot h_i h_j \right] \\
                   & = f(\ux) + \prods{\nabla f(\ux)}{\uh} + \frac{1}{2}\prods{Hf(\ux+\theta\uh) \cdot \uh}{\uh}
    \end{align*}
}
\begin{proof}
    Come per il caso precedente, la tesi viene da Taylor per \(k=2\):

    \[
        F(1) = F(0) + F'(0) + \frac{F''(\theta)}{2}
    \]

\end{proof}

\pagebreak
\subsubsection{Polinomio di Taylor con resto di Peano}

\defn{Resto di Peano in \(\Rn \)}{
    Date tutte le definizione iniziali;
    Se \(f \in C^2\), allora:

    \[
        f(\ux+\uh) = f(\ux) + \langle \nabla f(\ux),\uh \rangle + \frac{1}{2} \langle Hf(\ux) \cdot \uh, \uh \rangle + o(\norm{\uh}^2)
    \]
}

\defn{Resto di Peano in \(\R^2\)}{
    Date tutte le definizione iniziali;

    Siano \(\ux_0 =(x_0,y_0)\), \(\uh  = (h,k)\);

    Sia \(f \in C^2(A)\) tale che esista \(Hf(x,y) ~\forall (x,y) \in A\), dove quindi \(Hf(x,y)\) è simmetrica.

    Se tutto questo, allora:
    \begin{align*}
        f(x_0 + h, y_0 + k) = & \ f(x_0,y_0)                                                                                                       \\
                              & + f_x(x_0,y_0) \cdot h + f_y(x_0,y_0) \cdot k                                                                      \\
                              & + \frac{1}{2} \left[ f_{xx}(x_0,y_0) \cdot h^{2} + 2f_{xy}(x_0,y_0) \cdot hk + f_{yy}(x_0,y_0) \cdot k^{2} \right] \\
                              & + o(h^{2}+k^{2})
    \end{align*}
}
Il pezzo tra parentesi quadre è il risultato del prodotto tra matrici e vettori:
\begin{align*}
    \begin{bmatrix}
        f_{xx} & f_{xy} \\
        f_{xy} & f_{yy}
    \end{bmatrix}
    \cdot
    \begin{bmatrix}
        h \\
        k
    \end{bmatrix}
    \cdot
    \begin{bmatrix}
        h & k
    \end{bmatrix}
     & =
    \begin{bmatrix}
        f_{xx} \cdot h + f_{xy} \cdot k \\
        f_{xy} \cdot h + f_{yy} \cdot k
    \end{bmatrix}
    \cdot
    \begin{bmatrix}
        h & k
    \end{bmatrix}                                                                       \\
     & = (f_{xx} \cdot h + f_{xy} \cdot k)(h+k) + (f_{xy} \cdot h + f_{yy} \cdot k)(h+k) \\
\end{align*}
Quindi, \(\boxed{P_2(\ux, \ux_0)}\) è il polinomio omogeneo in due variabili, di grado 2, che meglio approssima la funzione \(f\) vicino a \(\ux_0\). Ovvero, per la definizione di o piccolo:
\[
    \lim_{\ux \to \ux_0} \frac{f(x,y) - P_2(x,y)}{\underbrace{{(x-x_0)}^2 + {(y-y_0)}^2}_{\norm{\ux-\ux_0}^2}} = 0
\]
In \(\ux_0 = (x_0,y_0)\) possiamo quindi approssimare la funzione come:

\[
    f(x,y) = P_2((x,y), (x_0, y_0)) + o({(x-x_0)}^{2} + {(y-y_0)}^{2})
\]
con:
\begin{align*}
    P_2(\ux, \ux_0) = & \ f(x_0,y_0) + f_x(x_0,y_0) (x-x_0) + f_y(x_0,y_0) (y-y_0)                                                                      \\
                      & + \frac{1}{2} \left[ [f_{xx}(x_0,y_0)]{(x-x_0)}^{2} + 2[f_{xy}(x_0,y_0)](x-x_0)(y-y_0) + [f_{yy}(x_0,y_0)]{(y-y_0)}^{2} \right]
\end{align*}
