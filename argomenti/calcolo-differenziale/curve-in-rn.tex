\subsection{Curve in \texorpdfstring{\(\Rn \)}{Rn}}

\subsubsection{Definizione di Curva in Rn}

Una curva è \textbf{un'applicazione continua} su un intervallo e così definita:
\begin{align*}
    \varphi: [a,b] \subset \R & \rightarrow \Rn                                                        \\
    t                         & \mapsto \varphi(t) = (\varphi_1(t),\varphi_2(t), \ldots ,\varphi_n(t))
\end{align*}
Le \textbf{equazioni parametriche} della curva sono:

\begin{equation*}
    \varphi:
    \begin{cases}
        \begin{aligned}
            x_1(t)      & = \varphi_1(t) \\
            x_2(t)      & = \varphi_2(t) \\
            \vdots\quad & = \quad\vdots  \\
            x_n(t)      & = \varphi_n(t)
        \end{aligned}
    \end{cases}
\end{equation*}

L'immagine della curva \(\varphi(I)\) è chiamata \textbf{sostegno della curva}:
\[
    \varphi(I) = \{\uy \in \Rn \giventhat \uy = \varphi(t) ~\forall t \in I\}
\]

\subsubsection{Curve piane (o cartesiane)}

In \(\R^2\) la curva si chiama \textbf{Curva Piana o Cartesiana}

Se poniamo:
\begin{align*}
     & \varphi: [a,b]\rightarrow \R^{2} \qquad \varphi: \begin{cases}
                                                            x = t \\
                                                            y = f(t)
                                                        \end{cases} \quad \forall t \in [a,b] \\[3mm]
     & t \mapsto \varphi(t) = (\varphi_1(t), \varphi_2(t)) = (t,f(t))
\end{align*}
avremmo che il sostegno della curva piana è il grafico di \(f\):
\[
    \graf{f} = \left\{(x,y) \in \R^2 \giventhat y=f(x) ~\forall x \in [a,b]\right\} = \left\{(x,f(x))\right\} \subset \R^2
\]

\subsubsection*{Esempi di curve cartesiane}

\(y=x+2 \quad \) per \(x \in [0,3]\):

Sostengo della curva: \( \{(x,x+2) \giventhat x \in [0,3]\} \)

\begin{equation*}
    \text{Equazioni parametriche}:
    \begin{cases}
        x = t \\
        y = t + 2
    \end{cases}
    \quad t \in [0,3]
\end{equation*}

\filbreak{}
\subsubsection{Curve regolari}

\defn{Arco di curva regolare}{
    Una curva si definisce \textbf{regolare} se:
    \begin{itemize}
        \item l'applicazione \(\varphi \) è di classe \(C^{1}\)

              ovvero, le derivate prime sono continue: \(\varphi_i : I \to \R \) è \(C^1(I)\)
        \item \(\varphi'(t) \neq 0 \quad\forall t \in (a,b)\)

              In particolare \(\varphi'(t) \neq 0\) significa che il vettore:
              \[
                  (\varphi_1'(t), \ldots ,\varphi_n'(t)) = \varphi'(t)
              \]
              non ha mai tutte le componenti contemporaneamente nulle.
              Quindi \(\norm{\varphi'(t)} > 0\)
    \end{itemize}
}

Si noti che \(\varphi'(t_0)\) è effettivamente un vettore. Inoltre, se \(\varphi(t)\) è regolare allora il vettore \(\varphi'(t_0)\) si chiama vettore tangente alla curva in \(\varphi(t_0) \in \Rn \)

NB{:} non tutte le curve sono regolari. Ad esempio, la curva definita da: \(\varphi(t) = (\cos 3t, \sin 3t)\), nonostante sia infinitamente derivabile, non è sempre non nulla. Infatti, per \(t=0\), \(\varphi'(t)=(0,0)\).

\subsubsection*{Esempio}

Nel caso di una curva piana cartesiana:
\begin{equation*}
    \varphi(t):
    \begin{cases}
        x = t \\
        y = f(t)
    \end{cases}
\end{equation*}
con \(f: \R \to \R \) una funzione tale che \(f \in C^1([a,b])\);

Fissato \(t_0 \in \R \), abbiamo che: \(\varphi'(t_0) = (1,f'(t_0))\).

Per cui la retta tangente al sostegno di \(\varphi \) risulta essere la retta tangente al grafico di \(f\) in \(t_0\).

\filbreak{}

\subsubsection*{Esempio di due curve con stesso sostegno}

Consideriamo le due applicazioni a valori vettoriali:

\[
    \underbrace{\varphi:[0,2\pi] \rightarrow \R^{2}}_{\varphi(t) = (\varphi_1(t),\varphi_2(t))}
\]

\[
    \underbrace{\psi: [0,4\pi] \rightarrow  \R^{2}}_{\psi(t) = (\psi_1(t),\psi_2(t))}
\]


\begin{equation*}
    \begin{cases}
        \varphi_1(t) = \cos t \\
        \varphi_2(t) = \sin t
    \end{cases}
\end{equation*}


\begin{equation*}
    \begin{cases}
        \psi_1(t) = \cos t \\
        \psi_2(t) = \sin t
    \end{cases}
\end{equation*}

Queste sono due curve piane che hanno lo stesso sostegno. Il sostegno di entrambe è dato dai punti del piano che stanno sulla circonferenza di centro \(\uzero \) e raggio \(1\), ma le due rimangono curve diverse.
