\subsection{Calcolo vettoriale}

\subsubsection{Vettori}

\defn{Vettore}{ Il vettore \(\ux \in \Rn \) è una n-upla \(\ux=(x_1,x_2, \ldots,x_n)\) }

In genere i vettori si indicano con \(\ux \) o \(\vec{x}\) o \(\overline{x}\), per indicare che sono variabili a più valori.

\subsubsection{Funzioni in più variabili}

In particolare:

\begin{itemize}
    \item \(f: A \in \Rn  \rightarrow \R \)
    \item \(g: A \in \Rn \rightarrow \R^{m}\)
\end{itemize}

Noi ci occuperemo solo di funzione reali di più variabili, ovvero funzioni da \(\Rn \to \R \)

\subsubsection*{Operazioni in \(\Rn \)}

\begin{itemize}
    \item Moltiplicazione per uno scalare reale:

          \[
              \forall c \in \R, \quad \forall \ux \in \Rn
          \]
          \[
              c \cdot \ux = (cx_1, \ldots ,cx_n)
          \]

    \item Somma:

          \[
              \forall \ux, \uy  \in \Rn
          \]
          \[
              \ux  =(x_1, \ldots ,x_n), \quad \uy  = (y_1, \ldots ,y_n)
          \]
          \[
              \ux+ \uy = (x_1+y_1, \ldots ,x_n+y_n)
          \]

\end{itemize}

\pagebreak{}

\subsubsection{Prodotto scalare}

L'operazione \(\prods{}{} : \Rn \times \Rn \rightarrow \R \)

\[
    \forall \ux ,\uy \in \Rn
\]
\[
    \prods{\ux}{\uy} := \ux \bullet \uy := \sum^{n}_{i=1} x_i y_i
\]

\subsubsection*{Esempio}
\[
    \ux=(1,2,0,3,5) \in \R^{5}
\]
\[
    \uy=(2,5,1,7,3) \in \R^{5}
\]
\[
    \prods{\ux}{\uy} = (1\cdot2)+(2\cdot5)+(0\cdot1)+(3\cdot7)+(5\cdot3) = 48
\]

\subsubsection*{Proprietà}

Il prodotto scalare verifica le seguenti proprietà

\begin{enumerate}
    \item \textbf{Bilinearità} (lineare su ogni fattore):

          \[
              (\alpha \ux_1 + \beta \ux_2) \bullet \uy = \prods{(\alpha \ux_1 + \beta \ux_2)}{\uy} = \alpha \prods{\ux_1}{\uy} + \beta \prods{\ux_2}{\uy}
          \]

          \[
              \forall \ux_1,\ux_2,\uy \in \R, \quad \forall \alpha, \beta \in \R
          \]

    \item \textbf{Simmetria} (l'ordine non conta)

          \[
              \forall \ux,\uy \in \Rn
          \]
          \[
              \prods{\ux}{\uy} = \prods{\uy}{\ux}
          \]

    \item \textbf{Positività}

          \[
              \forall \ux \in  \Rn
          \]
          \[
              \ux=(x_1, \ldots ,x_n)
          \]
          \[
              \prods{\ux}{\ux} = \sum^{n}_{i=1} {x_i}^{2}\ge 0
          \]
          \[
              \prods{\ux}{\ux} = 0 \iff \ux = \uzero = (0, \ldots ,0)\ \text{vettore nullo}
          \]

\end{enumerate}

\filbreak{}

\subsubsection{Norma di un vettore e proprietà associate}

\defn{Norma}{
    Il numero reale (non negativo)

    \[
        \norm\ux := \sqrt{\prods{\ux}{\ux}} := \sqrt{\sum_{i=1}^{n} {(x_i)}^2} \qquad (\ge 0)
    \]

    si chiama \textbf{lunghezza} o \textbf{norma} del vettore.

    Si può trovare anche scritta come \(|\ux|\), ma questa notazione è sconsigliata in quanto collide con quella di valore assoluto e quella di determinante.
}

\proposizione{Formula di Carnot}{

    \[
        \forall \ux,\uy \in \Rn
    \]
    \[
        \norm{\ux+\uy}^{2} = \norm\ux^{2} + \norm\uy^{2} + 2\prods{\ux}{\uy}
    \]
}

\begin{proof}
    \begin{align*}
        \norm{\ux+\uy} ^{2}  & = \langle \ux+\uy , \ux+\uy \rangle                                                       \\
        \text{(bilinearità)} & = \langle \ux,\ux+\uy \rangle + \langle \uy,\ux+\uy \rangle                               \\
        \text{(bilinearità)} & = \prods{\ux}{\ux} + \prods{\ux}{\uy} + \langle \uy,\ux \rangle + \langle \uy,\uy \rangle \\
                             & = \prods{\ux}{\ux} + 2\prods{\ux}{\uy} + \langle \uy,\uy \rangle                          \\
                             & = \norm\ux^{2} + \norm\uy^{2} + 2 \langle\ux, \uy\rangle
    \end{align*}

\end{proof}

\subsubsection*{Osservazione}

\(
\norm{\ux+\uy}^{2}= \norm\ux^{2} + \norm\uy^{2} \iff \langle \ux, \uy \rangle  =0
\)

\proposizione{Disuguaglianza di Cauchy-Schwarz}{
    \[
        \forall \ux,\uy \in \Rn
    \]
    \begin{align*}
        \circled{1} \qquad & |\prods{\ux}{\uy}| \le \norm\ux\cdot \norm\uy    &                                                                         \\
        \circled{2} \qquad & |\prods{\ux}{\uy}| = \norm\ux\cdot \norm\uy \iff &   & \uy = 0 \lor \ux=\lambda \uy \quad \text{ con } \lambda \in \R\ge 0 \\
                           &                                                  &   & \text{se e solo se un vettore è multiplo scalare dell'altro}
    \end{align*}
}

Vediamo ora le dimostrazioni di queste due proprietà:

\begin{proof} di \(\circled{1}\)

    Se \(\uy = \uzero = (0, \ldots ,0)\) la tesi è ovvia;

    Sia dunque \(\Rn  \rightarrow \uy \neq 0\) e consideriamo la funzione reale di una variabile reale \(t \rightarrow \norm{\ux+t\uy}^{2}\ge 0\) polinomio di secondo grado in t

    \begin{align*}
        ||\ux + t\uy|| ^{2} & \overset{\text{Carnot}}{=} \norm\ux^{2} + ||t\uy||^{2} + 2\langle \ux,t\uy \rangle \\
                            & ~~\, = ~~\, \norm\ux^{2} + \norm\uy^{2}t^{2} + 2 \prods{\ux}{\uy}t \ge 0
    \end{align*}

    è un polinomio di II grado in t dove \(\norm\uy^{2}> 0 \) essendo \(\uy \neq 0\) per ipotesi

    Il nostro \(\frac{\Delta}{4}\) (formula ridotta) deve essere negativo, altrimenti, in qualche punto il polinomio dovrebbe essere negativo ma la funzione di partenza era positiva:

    \begin{align*}
        \frac{\Delta}{4} = {\prods{\ux}{\uy}}^{2} - \norm\ux^{2}\norm\uy^{2} & \le 0                                       \\
        {\prods{\ux}{\uy}}^{2}                                               & \le \norm\ux^{2}\norm\uy^{2}                \\
        \prods{\ux}{\uy}                                                     & \le \norm\ux \norm\uy \quad (= \text{tesi})
    \end{align*}

\end{proof}

\filbreak{}

\begin{proof} di \(\circled{2}\)

    Se \(\uy = \uzero = (0, \ldots ,0)\) la testi è ovvia;

    Sia dunque \(\Rn  \rightarrow \uy \neq 0\) e consideriamo la funzione reale di una variabile reale \(t \rightarrow \norm{\ux+t\uy}^{2}\ge 0\) polinomio di secondo grado in t.
    La funzione ha \(\frac{\Delta}{4} = {\prods{\ux}{\uy}}^{2} - \norm\ux^{2}\norm\uy^{2}\)

    \medskip

    Se vale
    \[|\prods{\ux}{\uy}| = \norm\ux \cdot \norm\uy \]
    allora il \(\frac{\Delta}{4}\) del trinomio di II grado è nullo e dunque \(\exists t \in \R \) per cui
    \[\norm{\ux+t\uy}^{2}=0 \implies \ux +t\uy=0 \implies \ux=-t\uy \]
    per cui ci basta mostrare che \(-t \ge 0\)
    \begin{align*}
        |\prods{\ux}{\uy}| & = \norm\ux \cdot \norm\uy \\
        |\prods{\ux}{\uy}| & = \norm{-t\uy} \cdot \norm\uy \\
        |\prods{\ux}{\uy}| & = -t\norm{\uy} \cdot \norm\uy \\
        |\prods{\ux}{\uy}| & = -t\norm\uy^{2} \\[2mm]
        \frac{|\prods{\ux}{\uy}|}{\norm\uy^{2}} & = -t \\[2mm]
        \frac{\norm\ux \cdot \norm\uy}{\norm\uy^{2}} & = -t
    \end{align*}

    si ricorda che \(\norm\uy>0\) essendo \uy{} non nullo, quindi possiamo dire che

    \[
        -t = \frac{\norm\ux \cdot \norm\uy}{\norm\uy^{2}} \ge 0
    \]

\end{proof}

\filbreak{}

\subsubsection{Funzione ``lunghezza'' di un vettore}

Definiamo ora la funzione \textbf{lunghezza} di un vettore.

\begin{align*}
    \text{len}(\ux): \ \Rn & \rightarrow \R_0^{+}            \\
    \ux                    & \mapsto \norm\ux \quad (\geq 0)
\end{align*}

La funzione lunghezza è una norma, ovvero, \(\forall \ux \in \Rn , \quad \forall \lambda \in \R \):

\begin{enumerate}
    \item \(\norm\ux \in \R, \quad \norm\ux \geq 0\)
    \item \(\norm\ux = 0 \iff \ux = 0\)
    \item \(\norm{\lambda \ux} = |\lambda| \cdot \norm\ux \) \hfill (omogeneità)
    \item \(\norm{\ux+\uy} \leq \norm\ux + \norm\uy \) \hfill (disuguaglianza triangolare)
\end{enumerate}

\defn{Disuguaglianza triangolare}{
    La disuguaglianza triangolare si definisce come:

    \[
        \norm{\ux+\uy} \le \norm\ux+\norm\uy
    \]

    se \(\norm{\ux+\uy} = \norm\ux + \norm\uy \implies \uy =0 \lor \ux= \lambda \uy \qquad \text{con}\ \lambda \ge 0\):

}

\begin{proof}
    Dimostriamo la disuguaglianza triangolare, consideriamo:

    \begin{align*}
        \norm{\ux+\uy}^{2} & = \prods{\ux+\uy}{\ux+\uy}                                                                             \\
                           & = \norm\ux^{2}+\norm\uy^{2} + 2 \prods{\ux}{\uy}                                                       \\
                           & \le \norm\ux^{2} + \norm\uy ^{2} + 2|\prods{\ux}{\uy}|                                                 \\
                           & \le \underbrace{\norm\ux ^{2} + \norm\uy^{2} + 2(\norm\ux \cdot \norm\uy)}_{{(\norm\ux+\norm\uy)}^{2}}
    \end{align*}

    estraendo e passando alle radici si ha

    \[
        \norm{\ux+\uy} \le \norm\ux+\norm\uy
    \]
\end{proof}

