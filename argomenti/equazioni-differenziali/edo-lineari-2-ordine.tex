
\subsection{Equazioni differenziali lineari del II ordine}

\subsubsection{Definizione e soluzioni}

Le EDO lineari del II ordine sono della seguente forma:

\[
    a_2(x)y''(x) + a_1(x) y'(x) + a_0(x) y(x) = f(x)
\]

con \(a_0,a_1,a_2,f\) funzioni continue su \(I \subset \R \)

\defn{Forma normale}{
    \[
        y''(x) + a(x) y'(x) + b(x)y(x) = f(x) \qquad \circled{1}
    \]
}

\defn{Omogenea associata}{
    \[
        y''(x) + a(x) y'(x) + b(x)y(x) = 0 \qquad \circled{2}
    \]
}

\defn{Integrale generale}{
    \[
        \int generale\,(1) = \int generale\,(2) + \int particolare\,(1)
    \]
}

\subsubsection*{Integrale generale dell'omogenea associata}

Le soluzioni di \(\circled{2}\) sono le \(y(x)\) t.c. \(L(y(x)) = 0\).

Siano \(y_1(x)\) e \(y_2(x)\) due soluzioni t.c. \(L(y_1(x)) = L(y_2(x)) = 0\),
allora si vede subito che \(c_1y_1 + c_2y_2\) è anch'essa soluzione.

Per avere tutte le soluzioni però questo non basta, ho bisogno anche che \(y_1\) e \(y_2\) siano \\
\underline{linearmente indipendenti} su \(I\) (\(\forall x \in I\)).

Se si verifica tutto ciò, allora \(y(x) = c_1y_1 + c_2y_2\) è soluzione di \(\circled{2}\ \ \forall x \in I\)

\begin{proof}
    Siano \(y_1, y_2\) due soluzioni di \(a_2(x)y''(x) + a_1(x) y'(x) + a_0(x) y(x) = 0\), dimostro che \(y(x) = c_1 y_1(x) + c_2 y_2(x)\) è anch'essa soluzione.

    Sappiamo che le soluzioni soddisfano l'equazione per definizione, infatti:

    \[
        a_2(x)y_1''(x) + a_1(x) y_1'(x) + a_0(x) y_1(x) = 0
    \]

    \[
        a_2(x)y_2''(x) + a_1(x) y_2'(x) + a_0(x) y_2(x) = 0
    \]

    Derivo la loro combinazione due volte:

    \[
        y'(x) = c_1 y_1'(x) + c_2 y_2'(x)
    \]

    \[
        y''(x) = c_1 y_1''(x) + c_2 y_2''(x)
    \]

    e sostituisco a \(\circled{2}\):

    \begin{align*}
         & a_2(x)y''(x)                           & + & a_1(x) y'(x)                        & + & a_0(x) y(x)                       & = 0 \\
         & a_2(x) [ c_1y_1''(x) + c_2 y_2 ''(x) ] & + & a_1(x)[ c_1 y_1'(x) + c_2 y_2'(x) ] & + & a_0(x) [ c_1 y_1(x) + c_2 y_2(x)] & = 0 \\
    \end{align*}

    raggruppo:

    \begin{align*}
         & c_1[a_2(x) y_1''(x) + a_1(x) y_1'(x) + a_0(x) y_1(x)] & + & c_2 [a_2(x) y_2''(x) + a_1(x) y_2'(x) + a_0(x) y_2(x)] & = 0 \\
         & c_1 y_1(x)                                            & + & c_2 y_2(x)                                             & = 0
    \end{align*}
\end{proof}

Dunque, se abbiamo due soluzioni \(y_1\) e \(y_2\), esse sono linearmente indipendenti, ovvero, il determinante Wronskiano della matrice, \(y_1(x)y_2'(x)-y_2(x)y_1'(x)\), è diverso da 0.

\subsubsection*{Integrale particolare}

Lo vedremo poi con il metodo di variazione delle costanti

\subsubsection{EDO lineari di II ordine a coefficienti costanti}

\defn{Forma normale}{
    \[
        ay''(x) + by'(x) + cy(x) = f(x) \qquad \circled{1}
    \]

    con \(a \ne 0,\ \ x \in I,\ \ f \in C^0(I),\ \ a,b,c \in \R \)
}

\defn{Omogenea associata}{
    \[
        ay''(x) + by'(x) + cy(x) = 0 \qquad \circled{2}
    \]
}

\subsubsection{Integrale generale dell'omogenea associata}

Per risolvere questo tipo di EDO, si parte l'integrale generale dell'omogenea associata, dal quale identifichiamo due soluzioni \(y_1(x), y_2(x)\).

Quindi, considerata l'EDO (2), abbiamo due casi in base ai coefficienti \(b,c\):

\begin{itemize}
    \item
          \(b=c=0\)

          La (2) diventa:
          \[
              ay''(x) = 0 \implies y''(x) = 0 \quad \forall x \in I
          \]

          quindi:

          \[
              y'(x) = c \quad c \in \R
          \]

          \[
              y(x) = c_1x+c_0 \quad c_1,c_0 \in \R
          \]
    \item
          \(b,c\) non contemporaneamente nulli

          Se invece \(b\) e \(c\) non sono contemporaneamente nulli, devo considerare la seguente equazione algebrica di secondo grado in \(\C \):

          \[
              p(\lambda) = a \lambda^{2}+b \lambda + c =0
          \]

          Questa equazione si chiama \underline{equazione caratteristica} di (2) e l'integrale generale di (2) lo otteniamo studiando proprio le soluzioni di questa equazione.
\end{itemize}

\subsubsection*{Equazione caratteristica}

\[
    p(\lambda) = a \lambda^{2}+b \lambda + c =0
\]

Per il Teorema fondamentale dell'algebra questa equazione ha sempre due soluzioni in \(\C \) ma non sempre le ha in \(\R \).

\proposizione{}{
    \[
        y(x) = e^{\lambda x}\ \text{è soluzione di (2)} \iff p(\lambda)=0
    \]

    ovvero, se e soltanto se, \(\lambda \) è radice dell'equazione caratteristica associata a (2)
}
\begin{proof}
    Indico con \(Ly\) l'equazione \(Ly= ay''+by'+cy\)

    Dimostro che \(y\) è soluzione di (2) \(\iff \) \(Ly=0\)

    Se considero \(y(x) = e ^{\lambda x}\)

    Devo dimostrare che:

    \[
        L(e ^{\lambda x}) = 0 \iff  p(\lambda) = 0
    \]

    Sostituisco \(e ^{\lambda x}\) a \(x\):

    \begin{align*}
        L(e ^{\lambda x}) & = a( e^{\lambda x})'' + b( e ^{\lambda x})' + c(e ^{\lambda x})              \\
                          & = a \lambda ^{2} e ^{\lambda x} + b \lambda e ^{\lambda x} + c e^{\lambda x} \\
                          & = e ^{\lambda x}(a \lambda ^{2}+ b \lambda+ c)
    \end{align*}
    dunque
    \[
        L( e ^{\lambda x}) = 0 \iff a \lambda ^{2}+ b \lambda +c = 0
    \]

\end{proof}

\subsubsection*{Radici del polinomio caratteristico}

Adesso che ho dimostrato l'esistenza delle radici, devo trovarle.

Di solito le soluzioni di secondo grado si scrivono

\[
    \lambda_{1,2} = \frac{-b \pm \sqrt{b ^{2}-4 ac}}{2a}
\]

Le soluzioni \(\lambda_1\) e \(\lambda_2\) sono soluzioni di (2), quindi sono \(e ^{\lambda_1x}\) e \(e ^{\lambda_2x}\).

Distinguiamo tre casi per le soluzioni:

\begin{enumerate}
    \item due soluzioni reali e distinte (\(\Delta >0\)) \(\implies \lambda_1, \lambda_2\)
    \item due soluzioni reali e coincidenti (\(\Delta = 0\)) \(\implies \lambda_1, \lambda_2\)
    \item due soluzioni complesse coniugate (\( \Delta <0\)) \(\implies \alpha+i\beta, \alpha-i\beta \)
\end{enumerate}

\begin{align*}
    \text{i.} \quad   & y_1(x) = e ^{\lambda_1x}               &  & y_2(x) = e ^{\lambda_2x}               &  & \text{con}\  \lambda_1, \lambda_2 \in \R  \quad \text{t.c.} \quad \lambda_1 \neq \lambda_2 \\
    \text{ii.} \quad  & y_1(x) = e ^{\lambda x}                &  & y_2(x) = xe ^{\lambda x}               &  & \text{con}\  \lambda =\lambda_1=\lambda_2 = - \frac{b}{2a} \in \R                          \\
    \text{iii.} \quad & y_1(x) = e ^{\alpha x} \cos{(\beta x)} &  & y_2(x) = e ^{\alpha x} \sin{(\beta x)}                                                                                                 \\
\end{align*}

\filbreak{}

\teorema{}{L'integrale generale dell'equazione omogenea \(a y''+by'+c=0\) è dato da:

    \[
        c_1 y_1(x) + c_2 y_2(x)
    \]

    al variare di \(c_1,c_2 \in \R \) dove \(y_1(x)\) e \(y_2(x)\) sono definite come sopra
}

\subsubsection*{Caso \(\Delta > 0\)}

\begin{proof}
    \circled{1} \(b^{2}-4ac >0\) con \(\lambda_1,\lambda_2\) soluzioni dell'equazioni di \(p(\lambda)=0\)

    scrivo la Wronskiana di \(y_1,y_2\):
    \[
        \begin{bmatrix}

            e ^{\lambda_1 x}          & e ^{\lambda_2 x}          \\
            \lambda_1e ^{\lambda_1 x} & \lambda_2e ^{\lambda_2 x} \\
        \end{bmatrix}
    \]
    che è diverso da zero quindi le soluzioni sono linearmente indipendenti

    sia ora \(y(x)\) una soluzione di (2):

    \[
        y(x) = e ^{\lambda_1 x}u(x)
    \]

    io devo determinare \(u(x)\) per poi dimostrare che \(y(x) = c_1e ^{\lambda_1 x}+c_2 e^{\lambda_2 x}\)

    Poiché \(y(x) = e ^{\lambda_1 x}u(x)\) è soluzione di (2) si ha derivando e sostituendo:

    \begin{align*}
        \textcolor{Red}{a\left( e^{\lambda_{1}x}u(x) \right)''} + \textcolor{Cyan}{b\left( e^{\lambda_{1}x}u(x) \right)'} + \textcolor{Magenta}{ce^{\lambda_{1}x}u(x)}                                                                                                                                 & =0  \\
        \textcolor{Red}{a\left( \lambda_{1}e^{\lambda_{1}x}u(x)+ e^{\lambda_{1}x}u'(x) \right)'} + \textcolor{Cyan}{b\left( \lambda_1e^{\lambda_{1}x}u(x) + e^{\lambda_{1}x}u'(x) \right)} + \textcolor{Magenta}{ce^{\lambda_{1}x}u(x)}                                                                & =0  \\
        \textcolor{Red}{a\left( \lambda_{1}e^{\lambda_{1}x}u(x)+ e^{\lambda_{1}x}u'(x) \right)'} + e^{\lambda_{1}x} \left[ \textcolor{Cyan}{b\left( \lambda_{1}u(x) + u'(x) \right)} + \textcolor{Magenta}{cu(x)} \right]                                                                              & =0  \\
        \textcolor{Red}{a\left( {\lambda_{1}}^{2}e^{\lambda_{1}x}u(x) + \lambda_{1}e^{\lambda_{1}x}u'(x) + \lambda_{1}e^{\lambda_{1}x}u'(x) + e^{\lambda_{1}x}u''(x) \right)} + e^{\lambda_{1}x} \left[ \textcolor{Cyan}{b\left( \lambda_{1}u(x) + u'(x) \right)} + \textcolor{Magenta}{cu(x)} \right] & =0  \\
        e^{\lambda_{1}x} \left[ \textcolor{Red}{a\left( {\lambda_{1}}^{2}u(x) + 2\lambda_{1}u'(x) + u''(x) \right)} + \textcolor{Cyan}{b\left( \lambda_1u(x) + u'(x) \right)} + \textcolor{Magenta}{cu(x)} \right]                                                                                     & = 0 \\
        e^{\lambda_{1}x} \left[ au''(x) + (2a\lambda_{1} + b)u'(x) + \underbrace{\left(a{\lambda_{1}}^{2} + b\lambda_{1} + c\right)}_\text{impongo che sia =0}u(x) \right]                                                                                                                             & = 0
    \end{align*}

    estraggo solo l'ultima parentesi e impongo che sia uguale a zero perché il resto è già zero

    \[
        au''(x) + (2a \lambda_1 + b) u'(x) = 0
    \]

    divido per a:

    \[
        u''(x) +\left( 2 \lambda_1 + \frac{b}{a} \right) u'(x) = 0
    \]

    sapendo che:

    \[
        a \lambda^{2} + b \lambda + c =0
    \]

    \[
        \lambda^{2} + \frac{b}{a} \lambda + \frac{c}{a} =0
    \]

    \[
        \lambda_1 + \lambda_2 = -\frac{b}{a}
    \]

    \[
        \lambda_1  \lambda_2 = \frac{c}{a}
    \]

    \[
        u''(x) + (2 \lambda_1 - \lambda_1 - \lambda_2)u'(x) = 0
    \]

    il meno per comodità:

    \[
        u''(x) - (\lambda_1 - \lambda_2)u'(x) = 0
    \]

    se adesso chiamo \(u'(x)=v(x)\) e \(v''(x) = u'(x)\) l'equazione diventa:

    \[
        v' -kv = 0
    \]

    risolvendo

    \[
        v(x) = ce ^{kx}
    \]

    \[
        v(x) = c e^{(\lambda_2 - \lambda_1)x}
    \]

    sostituendo:

    \[
        u'(x) = c e ^{(\lambda_2- \lambda_1)x}
    \]

    integrando:

    \[
        u(x)  = c_1 e ^{(\lambda_2 - \lambda_1)x}+c_2
    \]

    la nostra \(y(x)\) diventa:

    \[
        y(x) = e ^{\lambda_1 x}u(x) = e ^{\lambda_1 x}( c_1 e ^{(\lambda_2 - \lambda_1)x}+c_2) = c_1 e ^{\lambda_2 x}+ c_2 e ^{\lambda_1 x}
    \]
\end{proof}

\filbreak{}

\subsubsection*{Caso \(\Delta = 0\)}

\begin{proof}
    Adesso voglio per il caso \circled{2}

    \[
        \lambda_1 = \lambda_2 = \lambda = -\frac{b}{2a} \in \R
    \]

    \[
        p(\lambda) =0 \iff e ^{\lambda x}\ \ \text{è soluzione di (2)}
    \]

    sia quindi \(y(x)\) una soluzione di (2) che scriviamo come:

    \[
        y(x) = e ^{\lambda x}u(x)
    \]

    Come prima si ottiene:

    \[
        a(e ^{\lambda x}u(x) )''+ b(e ^{\lambda x}u(x))' + c e ^{\lambda x}u(x)=0
    \]

    \[
        \overbrace{e ^{\lambda x}}^{>0}(a u''(x) + \underbrace{(a \lambda^{2}+b \lambda +c )}_\text{=0} u(x) + (2a \lambda+b)u'(x))=0
    \]

    estraggo la parte che impongo a zero:

    \[
        au''(x) + (2a \lambda+b)u'(x) = 0
    \]

    divido per a:

    \[
        u''(x) + \left( 2 \lambda + \frac{b}{a} \right)u'(x) = 0
    \]

    sapendo che \(-\frac{b}{a} = 2 \lambda \):

    \[
        u''(x) + \left( \cancel{2 \lambda} + \cancel{\frac{b}{a}} \right)u'(x) = 0
    \]


    \[
        u'(x) = c_1
    \]

    \[
        u(x) = c_1 x +c_2
    \]

    e quindi ho la soluzione:

    \[
        y(x) = e ^{\lambda x}(c_1x+c_2) = c_1x e^{\lambda x}+ c_2 e ^{\lambda x}
    \]
\end{proof}

\filbreak{}

\subsubsection*{Caso \(\Delta < 0\)}

TODO

\filbreak{}
\subsubsection{Soluzione particolare}

Una volta aver trovato l'integrale generale dell'omogenea dobbiamo vedere come si determina la soluzione particolare di:

\[
    ay''+by'+cy = f(x)\ \in I=[a,b]
\]

Indicheremo con \(\bar{y}\) la soluzione particolare. Per calcolarla ci sono 3 modi:

\begin{itemize}
    \item Caso elementare: \(b =c = 0\) \\
          \(\implies y'' = f(x)\) \\
          In questo caso trovare la soluzione è immediato, basta integrare due volte \(y''(x)\)
    \item Per similitudine: \\
          Si guarda il termine noto \(f(x)\) e si controlla se è in alcune forme particolari di cui sappiamo dire che la \(y(x)\) avrà una forma simile.
          In questo caso si procede usando la tabella apposita (vedi \underline{\nameref{sec:cheatsheet-edo-2-ordine-soluzione-particolare}}).
    \item Metodo di variazione delle costanti: \\
          Impongo che \(\bar{y}\) sia della seguente forma:
          \[
              \bar{y} = c_1(x) y_1(x) + c_2(x) y_2(x)
          \]
          dove \({y_1(x),y_2(x)}\) sono soluzioni linearmente indipendenti di (2), e \(c_1(x),c_2(x)\) sono funzioni di classe \(C^{2}(I)\) da determinare.
\end{itemize}

\subsubsection*{Metodo di variazione delle costanti}

L'obiettivo di questo metodo, è quello di trovare dei valori per \(c_1(x), c_2(x)\) avendo solamente conoscenza di \(y_1(x), y_2(x), f(x)\).

Mentre gli altri metodi sono sicuramente più facili e veloci, sono applicabili solo in casi particolari. Quando non è possibile utilizzarli bisogna procedere con il seguente metodo che è più complesso, ma è sempre applicabile.

Iniziamo; poiché \(\bar{y} (x)\) è soluzione di (1), possiamo dire che:
\begin{align*}
    \intertext{ \qquad \qquad \(a\bar{y}''(x) + b\bar{y}'(x) + c\bar{y}(x) = f(x) \quad \text{con} \quad \bar{y} (x) = c_1(x) y_1(x) + c_2(x) y_2(x)\) }
    \intertext{deriviamo \(\bar{y}\):}
    \bar{y} (x)   & = c_1(x) y_1(x) + c_2(x) y_2(x)                                         \\
    \bar{y} '(x)  & = c_1'(x) y_1(x) + c_1 y_1'(x) + c_2'(x) y_2(x) + c_2(x) y_2'(x)        \\
    \intertext{impongo che \(c_1'(x) y_1(x) + c_2'(x) y_2(x) = 0\) (prima condizione)}
    \bar{y} '(x)  & = c_1(x) y_1'(x) +c_2(x) y_2'(x)
    \intertext{deriviamo nuovamente:}
    \bar{y} ''(x) & = c_1'(x) y_1'(x) + c_1(x) y_1''(x) + c_2'(x) y_2'(x) + c_2(x) y_2''(x) \\
\end{align*}
Sostituisco le derivate nell'espressione iniziale:
\begin{align*}
    f(x) = & \ \ \textbf{a}\left[c_1'(x) y_1'(x) + c_1(x) y_1''(x) + c_2'(x) y_2'(x) +c_2(x) y_2''(x)\right] \\
           & + \textbf{b}\left[c_1(x) y_1'(x) + c_2(x) y_2'(x)\right]                                        \\
           & + \textbf{c}\left[c_1(x) y_1(x) + c_2(x) y_2(x)\right]
\end{align*}

Adesso raccolgo a fattore comune le \(c_i\):

\[
    c_1(x) [ \underbrace{a y_1''(x) +b y_1'(x) + c y_1(x)}_\text{=0}] + c_2(x) [\underbrace{a y_2''(x) + b y_2'(x) + c y_2(x)}_\text{=0}]+ a [c_1'(x) y_1'(x)+ c_2'(x) y_2'(x)] = f(x)
\]

quindi mi rimane:

\[
    c_1'(x) y_1'(x) + c_2'(x) y_2'(x) = \frac{f(x)}{a}
\]

Ottengo il sistema di 2 equazioni nelle due incognite (\(c_1'(x),c_2'(x)\)) non omogeneo:

\begin{equation*}
    \begin{cases}
        c_1'(x)y_1(x) + c_2'(x) y_2(x) = 0 \\
        c_1'(x) y_1'(x) + c_2'(x) y_2'(x)  = \frac{f(x)}{a}
    \end{cases}
\end{equation*}

La matrice dei coefficienti del sistema è:

\[
    \begin{bmatrix}
        y_1(x)  & y_2(x)  \\
        y_1'(x) & y_2'(x) \\
    \end{bmatrix}
    \neq 0
\]

È la matrice Wronskiana.

Uso il metodo di Cramer per risolvere il sistema:

\[
    A = \begin{pmatrix}
        a_{11} & a_{12} \\
        a_{21} & a_{22} \\
    \end{pmatrix}
\]

\(\det A = a_{11} a_{22} - a_{12} a_{21}\)

Il metodo:

\[
    c_1'(x) =
    \frac{
        \begin{vmatrix}
            0              & y_2(x)  \\
            \frac{f(x)}{a} & y_2'(x) \\
        \end{vmatrix}
    }{
        \begin{vmatrix}
            y_1(x)  & y_2(x)  \\
            y_1'(x) & y_2'(x) \\
        \end{vmatrix}
    } = \frac{- y_2(x) \frac{f(x)}{a}}{y_1(x) y_2'(x) - y_2(x) y_1'(x)}
\]

\[
    c_2'(x) =
    \frac{
        \begin{vmatrix}
            y_1(x)  & 0              \\
            y_1'(x) & \frac{f(x)}{a} \\
        \end{vmatrix}
    }{
        \begin{vmatrix}
            y_1(x)  & y_2(x)  \\
            y_1'(x) & y_2'(x) \\
        \end{vmatrix}
    } = \frac{- y_1(x) \frac{f(x)}{a}}{y_1(x) y_2'(x) - y_2(x) y_1'(x)}
\]

\filbreak{}

Ora dobbiamo integrare

\[
    c_1(x) = \int c_1'(x) \diff x
\]
\[
    c_2(x) = \int c_2'(x) \diff x
\]

Quindi alla fine della fiera abbiamo un modo per trovare questi \(c_1(x), c_2(x)\), e possiamo scrivere la soluzione particolare come:

\[ \bar{y} = c_1(x)y_1(x) + c_2(x)y_2(x)\]

\filbreak{}

\subsubsection{Forma dell'integrale generale}

Alla fine si ha quindi che \underline{l'integrale generale} di (1) è nella seguente forma:

\[
    y(t)  = \underbrace{c_1 y_1(x) + c_2 y_2(x)}_\text{generale} + \underbrace{c_3(x) y_1(x) + c_4(x) y_2(x)}_{\text{particolare}\ =\ \bar{y}(x)}
\]

\subsubsection{Problema di Cauchy}

Assegnando le condizioni iniziali:

\[
    y(x_0) = y_0
\]
\[
    y'(x_0) = y_1
\]

Quindi il problema di Cauchy mi viene:

\begin{equation*}
    \begin{cases*}
        ay''+ by'+cy= f(x) \\
        y(x_0)=y_0         \\
        y'(x_0) = y_1
    \end{cases*}
\end{equation*}

\pagebreak{}

\subsubsection{Esercizi}

Trovare le soluzioni dei seguenti problemi di Cauchy

\subsubsection*{Esempio 1 (a occhio)}

\begin{equation*}
    \begin{cases*}
        3y'' + 5y' + 2y=3e ^{2x} \\
        y(0) = 0                 \\
        y'(0) = 1
    \end{cases*}
\end{equation*}

Scriviamo (1) e (2):

\begin{align*}
    \circled{1} & \qquad 3y''+5y'+2y = 3 e^{2x} \\
    \circled{2} & \qquad 3y''+5y'+2y = 0
\end{align*}

Risolvo:

\[
    3 \lambda ^{2} + 5 \lambda + 2 = 0
\]

\[
    3 \lambda^{2} + 3 \lambda + 2 \lambda + 2 =0
\]

\[
    3 \lambda ( \lambda+1) + 2( \lambda + 1) =0
\]

\[
    (3 \lambda +2 ) ( \lambda +1 ) =0
\]

\[
    \lambda= -\frac{2}{3}, \lambda=-1
\]

quindi ho soluzioni linearmente indipendenti di (2):

\[
    y_1(x) = e ^{-\frac{2}{3}x}, y_2(x) = e ^{-x}
\]

Integrale generale di (2):

\[
    y_0(x) = c_1 e^{-\frac{2}{3}x} + c_2 e ^{-x} \quad c_1,c_2 \in \R
\]


Una soluzione particolare di (1) è dunque:

\[
    \bar{y} (x) = A e ^{2x}
\]

Ora derivo due volte:

\[
    \bar{y} '(x) = 2A e ^{2x}
\]

\[
    \bar{y} ''(x) = 4A e ^{2x}
\]

sostituendo poi in (1):

\[
    12Ae ^{2x} + 10 A e ^{2x} + 2 A e ^{2x} = 3 e^{2x}
\]

\[
    24A e ^{2x} = 3 e ^{2x}
\]

\[
    A = \frac{1}{8}
\]

Adesso devo imporre le condizioni iniziali a queste due:

\[
    y(x) = c_1 e ^{-\frac{2}{3}x} + c_2 e ^{-x}+ \frac{1}{8}e ^{2x}
\]

\[
    y'(x) = -\frac{2}{3}c_1 e ^{-\frac{2}{3}x} - c_2 e ^{-x}+ \frac{1}{4} e ^{2x}
\]


\begin{equation*}
    \begin{cases*}
        c_1+c_2+\frac{1}{8}=0 \\
        -\frac{2}{3}c_1-c_2+\frac{1}{4}=1
    \end{cases*}
\end{equation*}

\begin{equation*}
    \begin{cases*}
        c_1=c_2-\frac{1}{8} \\
        \frac{2}{3}c_2+\frac{1}{12}-c_2+\frac{1}{4}=1
    \end{cases*}
\end{equation*}


\begin{equation*}
    \begin{cases*}
        c_1=c_2-\frac{1}{8} \\
        -\frac{1}{3}c_2= 1- \frac{1}{3}
    \end{cases*}
\end{equation*}

\begin{equation*}
    \begin{cases*}
        c_1=c_2-\frac{1}{8} \\
        -\frac{1}{3}c_2=\frac{2}{3}
    \end{cases*}
\end{equation*}

\begin{equation*}
    \begin{cases*}
        c_1=\frac{15}{8} \\
        c_2=-2
    \end{cases*}
\end{equation*}

La soluzione del problema è quindi:

\[
    y(x) = \frac{15}{8}e ^{-\frac{2}{3}x}- 2 e ^{-x}+ \frac{1}{8}e ^{2x}
\]

\filbreak{}
\subsubsection*{Esempio 2}

\begin{equation*}
    \begin{cases*}
        y''(x) -2y'(x)+ y(x) = 5\sin(x) \\
        y(0) = 0                        \\
        y'(0) = 1
    \end{cases*}
\end{equation*}


Risolvo l'equazione caratteristica associata:

\[
    \lambda^{2}-2 \lambda + 1=0
\]

\[
    {(\lambda-1)}^{2}=0
\]

quindi \(\lambda_1=\lambda_2=1\). Dunque il nostro integrale generale di (2) è:

\[
    z(x) = c_1 e ^{x} + c_2 x e ^{x} \quad c_1,c_2 \in \R
\]

Per trovare la soluzione particolare usiamo il metodo di somiglianza per il termine noto:

\[
    \bar{y} (x) = A \cos(x) + B \sin(x)
\]

le derivate:

\[
    \bar{y} '(x)  = -A \sin(x) + B \cos(x)
\]

\[
    \bar{y} ''(x)  = -A \cos(x) - B \sin(x)
\]


Sostituiamo a quella iniziale:

\[
    \cancel{-A \cos(x)} \cancel{- B\sin(x)} + 2A \sin(x) - 2B \cos(x)+ \cancel{A \cos(x)} + \cancel{B\sin(x)} = 5\sin(x)
\]

e quindi \(2A = 5\) e \(B=0\):

\[
    \bar{y} (x) = \frac{5}{2} \cos(x)
\]

adesso devo trovare:

\[
    y(x) = c_1 e ^{x}+ c_2 x e ^{x} + \frac{5}{2} \cos(x)
\]

\[
    y'(x)  = c_1 e ^{x} + c_2 e ^{x} + c_2x e ^{x} - \frac{5}{2}\sin(x)
\]

Adesso impongo le condizioni iniziali:

\begin{equation*}
    \begin{cases*}
        c_1 +\frac{5}{2}=0 \\
        c_1+c_2= 1
    \end{cases*}
\end{equation*}

\begin{equation*}
    \begin{cases*}
        c_1 = -\frac{5}{2} \\
        c_2 = \frac{7}{2}
    \end{cases*}
\end{equation*}

La soluzione del problema è quindi:

\[
    y(x) = -\frac{5}{2}e ^{x}+ \frac{7}{2} x e ^{x} + \frac{5}{2} \cos(x)
\]

\filbreak{}
\subsubsection*{Esempio 3}

\begin{equation*}
    \begin{cases*}
        y''+ 2y'+ 2y= 3x^{2} \\
        y(0) = 0             \\
        y'(0) = 1
    \end{cases*}
\end{equation*}


risolvo:

\[
    \lambda^{2}+2 \lambda+2 =0
\]

\[
    \lambda= \frac{-1 \pm \sqrt{1-2}}{1}= -1 \pm i
\]

le soluzioni sono complesse:

\[
    \alpha \pm  i \beta
\]

con \(\alpha = -1\) e \(\beta = 1\) quindi sostituisco:

\[
    y_0(x) = y_0(c_1,c_2) = e ^{-x}(c_1\cos(x) + c_2 \sin(x))
\]

la soluzione particolare:

\[
    \bar{y} (x) = A x^{2}+Bx+C
\]

derivo due volte:

\[
    \bar{y} '(x) = 2Ax + B
\]

\[
    \bar{y} ''(x) = 2A
\]

sostituisco in (1):

\[
    2A + 4Ax + 2B + 2Ax^{2}+2Bx + 2C = 3x^{2}
\]

\[
    2Ax^{2} + 2(2A+B) x + 2(A+B+C) = 3x^{2}
\]

risolvo un sistema per le incognite \(A,B,C\) e quindi trovo che:

\begin{equation*}
    \begin{cases*}
        2A = 3     \\
        2A + B = 0 \\
        A+B+C= 0
    \end{cases*}
\end{equation*}

\begin{equation*}
    \begin{cases*}
        A=\frac{3}{2} \\
        B=-3          \\
        \frac{3}{2}-3+C=0
    \end{cases*}
\end{equation*}

\begin{equation*}
    \begin{cases*}
        A=\frac{3}{2} \\
        B=-3          \\
        C=\frac{3}{2}
    \end{cases*}
\end{equation*}


Quindi la soluzione particolare:

\[
    \bar{y} (x) = \frac{3}{2}x^{2}-3x+\frac{3}{2}
\]


l'espressione quindi è:

\[
    y(x) = e ^{-x}(c_1 \cos(x) + c_2 \sin(x) ) + \frac{3}{2}x^{2}-3x+\frac{3}{2}
\]

derivo:

\[
    y'(x) = e ^{-x}(c_1 \cos(x)+ c_2 \sin(x)) + e ^{-x}(-c_1\sin(x)+c_2 \cos(x)) +3x -3
\]

quindi impongo le condizioni:

\begin{equation*}
    \begin{cases*}
        c_1 +\frac{3}{2}=0 \\
        -c_1+c_2 -3 = 1
    \end{cases*}
\end{equation*}

\begin{equation*}
    \begin{cases*}
        c_1=-\frac{3}{2} \\
        c_2=-\frac{5}{2}
    \end{cases*}
\end{equation*}

La soluzione del problema è quindi:

\[
    y(x) = e ^{-x}(-\frac{3}{2}\cos(x) + \frac{5}{2} \sin(x)) + \frac{3}{2}x^{2}-3x + \frac{3}{2}
\]

\filbreak{}
\subsubsection*{Esempio 4}

\begin{equation*}
    \begin{cases*}
        y''-6y' + 9y = e ^{3x} \\
        y(0) = 1               \\
        y'(0) = 0
    \end{cases*}
\end{equation*}

\[
    \lambda^{2} -6 \lambda + 9 = 0
\]

\[
    \lambda_1=\lambda_2=3
\]

La soluzione generale dell'omogenea associata è:
\[
    z(x)  = c_1 e ^{3x}+ c_2 x e ^{3x}
\]

Per la soluzione particolare usiamo il metodo di somiglianza per il termine noto esponenziale. Notiamo che nel termine noto abbiamo nell'esponente \(\lambda=3\) che è anche soluzione dell'equazione caratteristica. La soluzione particolare è quindi:

\[
    \bar{y} (x) = Ax^{2}e^{3x}
\]

dove il grado della \(x\) viene dalla molteplicità di lambda come soluzione del polinomio caratteristico. Procediamo a trovare \(A\):

\[
    \bar{y}'(x) = 2Axe^{3x} + 3Ax^{2}e^{3x}
\]
\[
    \bar{y}''(x) = 2Ae^{3x} + 6Axe^{3x} + 6Axe^{3x} + 9Ax^2e^{3x} = 2Ae^{3x} + 12Axe^{3x} + 9Ax^2e^{3x}
\]
Sostituiamo nella EDO e troviamo \(A\):
\begin{align*}
    \left[ 2Ae^{3x} + 12Axe^{3x} + 9Ax^2e^{3x} \right] -6\left[ 2Axe^{3x} + 3Ax^{2}e^{3x} \right] + 9 \left[ Ax^{2}e^{3x} \right] & = e^{3x}      \\
    2Ae^{3x}                                                                                                                      & = e^{3x}      \\
    A                                                                                                                             & = \frac{1}{2}
\end{align*}

La soluzione generale della EDO è quindi:

\[
    y(x) = c_1 e ^{3x}+ c_2 x e ^{3x} + \frac{1}{2} x^{2}e ^{3x}
\]

Impongo ora le condizioni iniziali:
\begin{align*}
     & y'(x) = 3c_1 e ^{3x}+ \left( c_2 e ^{3x} + 3c_2 x e ^{3x} \right) + \left( xe ^{3x} + \frac{3}{2} x^{2}e ^{3x} \right) \\
     & y(0) = c_1 = 1                                                                                                         \\
     & y'(0) = 3c_1 + c_2 = 0
\end{align*}
Quindi, \(c_1 = 1\) e \(c_2 = -3\). Dunque, la soluzione al problema di Cauchy è:

\[
    y(x) = e^{3x} -3xe^{3x} +\frac{1}{2}x^{2}e^{3x}
\]

\filbreak{}
\subsubsection*{Esempio 5}

\begin{equation*}
    \begin{cases*}
        y'= \frac{\sin(x)}{\cos(y)} \\
        y\left( \frac{\pi}{2} \right) = \frac{\pi}{6}
    \end{cases*}
\end{equation*}

questa è a variabili separabili:

\[
    y'(x)\cdot\cos(y) = \sin(x)
\]

\[
    \int y'(x)\cdot\cos(y) \diff x = \int \sin(x) \diff x
\]

\[
    \int \cos(y) \diff y = \int \sin(x) \diff x
\]

\[
    \sin(y) = -\cos(x) +c
\]

impongo adesso la condizione iniziale (\(x=\frac{\pi}{2}\)):

\begin{align*}
    \sin\left(\frac{\pi}{6}\right) & = -\cos\left(\frac{\pi}{2}\right) + c \\ \\
    \sin\left(\frac{\pi}{6}\right) & = -0 +c                               \\ \\
    \frac{1}{2}                    & = c
\end{align*}

quindi sostituisco la c:

\[
    \sin\left(y\right) = - \cos(x) + \frac{1}{2}
\]

Per ottenere \(y\) dovrò fare l'arcoseno, quindi guardo dove è definita l'espressione:

\[
    -1 \le \sin\left(y\right) \le 1
\]

ovvero:

\[
    -1 \le \cos(x) + \frac{1}{2} \le 1
\]

impongo il sistema:

\begin{equation*}
    \begin{cases*}
        -\cos(x) +\frac{1}{2}\ge -1 \\
        -\cos(x) + \frac{1}{2} \le 1
    \end{cases*}
\end{equation*}

\begin{equation*}
    \begin{cases*}
        \cos(x) \le  \frac{3}{2} \\
        \cos(x) \ge -\frac{1}{2}
    \end{cases*}
\end{equation*}

Prendo la soluzione nell'intervallo:

\[
    -\frac{2}{3}\pi \le  x \le  \frac{2}{3}\pi
\]

la soluzione del problema è quindi:

\[
    y(x) = \arcsin \left( -\cos(x) +\frac{1}{2} \right)
\]

con la \(x\) presa nell'intervallo appena definito.

\pagebreak
\subsubsection{Cheatsheet {-} Soluzione generale dell'omogenea associata}

Cheatsheet su come trovare la soluzione generale dell'omogenea associata a una EDO lineare del 2 ordine non omogenea a coefficienti costanti.

Ovvero, le soluzioni di una EDO della forma \(ay''+by'+cy=0\) con \(a,b,c \in \R \) e \(a \ne 0\).

\begin{itemize}
    \item \(b=c=0\):

          \(y(x) = c_{2}x + c_{1}\)

    \item \(b \ne c\):

          Si guarda l'equazione caratteristica associata \(a\lambda^2 + b\lambda + c =0\):
          \begin{itemize}
              \item \(\Delta > 0\) due soluzioni distinte \(\lambda_1 \ne \lambda_2\):

                    \smallskip
                    \(y(x) = c_{1}e^{\lambda_{1}x} + c_{2}e^{\lambda_{2}x}\)
                    \medskip
              \item \(\Delta = 0\) due soluzioni coincidenti \(\lambda_1=\lambda_2\):

                    \smallskip
                    \(y(x) = c_{1}e^{\lambda_{1}x} + xc_{2}e^{\lambda_{2}x}\)
                    \medskip
              \item \(\Delta < 0\) due soluzioni complesse coniugate \(\alpha \pm i\beta \):

                    \smallskip
                    \(y(x) = c_{1}e^{\alpha x}\cos(\beta x) + c_{2}e^{\alpha x}\sin(\beta x)\)
                    \medskip
          \end{itemize}
\end{itemize}

\pagebreak
\subsubsection{Cheatsheet {-} Metodo di somiglianza per la soluzione particolare}\label{sec:cheatsheet-edo-2-ordine-soluzione-particolare}

Cheatsheet sul \textbf{metodo di somiglianza} utile per trovare la soluzione particolare di EDO lineari del 2 ordine non omogenee a coefficienti costanti in dei casi noti.

Ovvero, per trovare la \(\bar{y}(x)\) di EDO nella forma \(ay''+by'+cy=f(x)\) con \(a,b,c \in \R \) e \(a \ne 0\), in cui \(f(x)\) assume delle forme note.

\begin{itemize}
    \item \(ay''+by'+cy = P_n[x]\)
          \(\quad \implies \) trovare \(A_{i} \in \R \) con:
          \begin{itemize}
              \item \(c \ne 0\)

                    \smallskip
                    \(\bar{y}(x) = A_0 + A_{1}x + \cdots + A_{n}x^{n}\)
                    \medskip
              \item \(c = 0, b \ne 0\)

                    \smallskip
                    \(\bar{y}(x) = x (A_0 + A_{1}x + \cdots + A_{n}x^{n})\)
                    \medskip
              \item \(c = 0, b = 0\)

                    \smallskip
                    \(\bar{y}(x) = x^{2} (A_0 + A_{1}x + \cdots + A_{n}x^{n})\)
                    \medskip
          \end{itemize}
    \item \(ay''+by'+cy = Ae^{\lambda x}\)
          \(\quad \implies \) trovare \(c_{1} \in \R \) con:
          \begin{itemize}
              \item \(\lambda = y\) non è soluzione dell'equazione caratteristica

                    \smallskip
                    \(\bar{y}(x) = c_{1}e^{\lambda x}\)
                    \medskip
              \item \(\lambda = y\) è soluzione dell'equazione caratteristica

                    \smallskip
                    \(\bar{y}(x) = c_{1}xe^{\lambda x}\)
                    \medskip
              \item \(\lambda = y\) è una soluzione di molteplicità \(m\) dell'equazione caratteristica

                    \smallskip
                    \(\bar{y}(x) = c_{1}x^{m}e^{\lambda x}\)
                    \medskip
          \end{itemize}
    \item \(ay''+by'+cy = A_{1}\cos(\omega x) + A_{2}\sin(\omega x)\)
          \(\quad \implies \) trovare \(c_1,c_2 \in \R \) con:
          \begin{itemize}
              \item \(b \ne 0\)

                    \smallskip
                    \(\bar{y}(x) = c_{1}\cos(\omega x) + c_{2}\sin(\omega x)\)
                    \medskip
              \item \(b = 0\ \text{ e }\ c_{1}\cos(\omega x) + c_{2}\sin(\omega x)\) è soluzione dell'omogenea associata alla EDO iniziale

                    \smallskip
                    \(\bar{y}(x) = x[c_{1}\cos(\omega x) + c_{2}\sin(\omega x)]\)
                    \medskip
          \end{itemize}
    \item \(ay''+by'+cy = P_n[x] \cdot e^{\lambda x}\)
          \(\quad \implies \) trovare \(A_{i} \in \R \) con:
          \begin{itemize}
              \item \(\lambda = y\) non è soluzione dell'equazione caratteristica

                    \smallskip
                    \(\bar{y}(x) = e^{\lambda x} (A_0 + A_{1}x + \cdots + A_{n}x^{n})\)
                    \medskip
              \item \(\lambda = y\) è soluzione di molteplicità \(m\) dell'equazione caratteristica

                    \smallskip
                    \(\bar{y}(x) = e^{\lambda x} x^{m} (A_0 + A_{1}x + \cdots + A_{n}x^{n})\)
                    \medskip
          \end{itemize}
    \item \(ay''+by'+cy = e^{\lambda x} \cdot [A_{1}\cos(\omega x) + A_{2}\sin(\omega x)]\)
          \(\quad \implies \) trovare \(c_1,c_2 \in \R \) con:
          \begin{itemize}
              \item \(z = \lambda + i\omega \) non è soluzione dell'equazione caratteristica

                    \smallskip
                    \(\bar{y}(x) = e^{\lambda x} [c_{1}\cos(\omega x) + c_{2}\sin(\omega x)]\)
                    \medskip
              \item \(z = \lambda + i\omega \) è soluzione dell'equazione caratteristica

                    \smallskip
                    \(\bar{y}(x) = x e^{\lambda x} [c_{1}\cos(\omega x) + c_{2}\sin(\omega x)]\)
          \end{itemize}
\end{itemize}

\filbreak{}
\subsubsection*{Osservazioni}

Se \(f(x) = f_1(x) + f_2(x)\) e le due funzioni \(f_1(x),f_2(x)\) sono forme note presenti nella tabella sopra, allora si cercano separatamente una soluzione particolare \(\bar{y}_1(x)\) dell'equazione \(ay''+by'+cy=f_1(x)\) e una soluzione particolare \(\bar{y}_2(x)\) dell'equazione \(ay''+by'+cy=f_2(x)\).

Dopodiché, per la linearità dell'equazione differenziale, si ha che la funzione \(\bar{y}(x) = \bar{y}_1(x) + \bar{y}_2(x)\) è soluzione particolare di \(ay''+by'+cy=f_1(x) + f_2(x)\)
