
\subsection{Equazioni differenziali lineari del II ordine}

\subsubsection{Definizione e soluzioni}

Le EDO lineari del II ordine sono della seguente forma:

\[
    a_2(x)y''(x) + a_1(x) y'(x) + a_0(x) y(x) = f(x)
\]

con \(a_0,a_1,a_2,f\) funzioni continue su \(I \subseteq \R \)

\defn{Forma normale}{
    \[
        y''(x) + a(x) y'(x) + b(x)y(x) = f(x) \qquad \circled{1}
    \]
}

\defn{Omogenea associata}{
    \[
        y''(x) + a(x) y'(x) + b(x)y(x) = 0 \qquad \circled{2}
    \]
}

\defn{Integrale generale}{
    \[
        \int generale\,(1) = \int generale\,(2) + \int particolare\,(1)
    \]
}

\pagebreak
\subsubsection{Integrale generale dell'omogenea associata}

Le soluzioni di \(\circled{2}\) sono le \(y(x)\) t.c. \(L(y(x)) = 0\).

Siano \(y_1(x)\) e \(y_2(x)\) due soluzioni t.c. \(L(y_1(x)) = L(y_2(x)) = 0\),
allora si vede subito che \(c_1y_1 + c_2y_2\) è anch'essa soluzione.

Per avere tutte le soluzioni però questo non basta, ho bisogno anche che \(y_1\) e \(y_2\) siano \\
\underline{linearmente indipendenti} su \(I\) (\(\forall x \in I\)).

Se si verifica tutto ciò, allora \(y(x) = c_1y_1 + c_2y_2\) è soluzione di \(\circled{2}\ \ \forall x \in I\)

\begin{proof}
    Siano \(y_1, y_2\) due soluzioni di \(a_2(x)y''(x) + a_1(x) y'(x) + a_0(x) y(x) = 0\), dimostro che \(y(x) = c_1 y_1(x) + c_2 y_2(x)\) è anch'essa soluzione.

    Sappiamo che le soluzioni soddisfano l'equazione per definizione, infatti:

    \[
        a_2(x)y_1''(x) + a_1(x) y_1'(x) + a_0(x) y_1(x) = 0
    \]

    \[
        a_2(x)y_2''(x) + a_1(x) y_2'(x) + a_0(x) y_2(x) = 0
    \]

    Derivo la loro combinazione due volte:

    \[
        y'(x) = c_1 y_1'(x) + c_2 y_2'(x)
    \]

    \[
        y''(x) = c_1 y_1''(x) + c_2 y_2''(x)
    \]

    e sostituisco a \(\circled{2}\):

    \begin{align*}
         & a_2(x)y''(x)                           & + & a_1(x) y'(x)                        & + & a_0(x) y(x)                       & = 0 \\
         & a_2(x) [ c_1y_1''(x) + c_2 y_2 ''(x) ] & + & a_1(x)[ c_1 y_1'(x) + c_2 y_2'(x) ] & + & a_0(x) [ c_1 y_1(x) + c_2 y_2(x)] & = 0 \\
    \end{align*}

    raggruppo:

    \begin{align*}
         & c_1[a_2(x) y_1''(x) + a_1(x) y_1'(x) + a_0(x) y_1(x)] & + & c_2 [a_2(x) y_2''(x) + a_1(x) y_2'(x) + a_0(x) y_2(x)] & = 0 \\
         & c_1 y_1(x)                                            & + & c_2 y_2(x)                                             & = 0
    \end{align*}
\end{proof}

Dunque, se abbiamo due soluzioni \(y_1\) e \(y_2\), esse sono linearmente indipendenti, ovvero, il determinante Wronskiano della matrice, \(y_1(x)y_2'(x)-y_2(x)y_1'(x)\), è diverso da 0.

\subsubsection{Integrale particolare}

Nel caso generico si può trovare usando il metodo di variazione delle costanti, chiamato anche metodo di Lagrange.

Al fine di questo corso, ci limiteremo a studiarne il funzionamento nel caso specifico delle EDO lineari del II ordine a coefficienti costanti trattate nel capitolo successivo.
