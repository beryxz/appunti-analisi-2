\subsection{Equazioni differenziali lineari}

\subsubsection{Forma lineare}

Una EDO di ordine \(n\) si dice \textbf{lineare} se è nella forma
\[
    a_n(x)y^{(n)}+a_{n-1}(x)y^{(n-1)}+ \cdots + a_2(x)y''+a_1(x)y'+a_0(x)y=f(x)
\]

Dove, \(x \in I\)\ \ e\ \  \(a_0(x),\ldots,a_n(x),\,f(x)\) sono funzioni \underline{assegnate} e \underline{continue} sull'intervallo I.

Ad esempio:
\[
    xy''+5y = \sin(x)
\]

\subsubsection{Forma omogenea}

Quando abbiamo \(0\) come \(f(x)\) allora l'equazione si dice \textbf{omogenea}.
\[
    a_n(x)y^{(n)}+a_{n-1}(x)y^{(n-1)}+ \cdots + a_2(x)y''+a_1(x)y'+a_0(x)y=0
\]

Inoltre, generalmente le EDO lineari si prendono o si trasformano per averle con \(a_{n}(x) = 1\).

\subsubsection{Origine della linearità}

Il termine \textbf{lineare} viene dal fatto che se consideriamo il primo membro di questo tipo di equazioni differenziali come legge di un applicazione, questa applicazione rispetta tutte le condizioni per essere definita lineare.

Indicando dunque con \(L(y(x))\) il primo membro dell'equazione differenziale,
\[
    L(y(x)) := a_n (x) y^{(n)} (x) + \cdots + a_1(x)y'(x) + a_0(x)y(x)
\]
possiamo notare che l'operatore \(L_y\), definito come:
\begin{align*}
     & L_y: C^{n}(I) \rightarrow C^0(I) \\
     & L_y: y(x) \mapsto L(y(x))
\end{align*}
risulta essere un operatore lineare tra questi spazi di funzioni. Infatti, nell'ipotesi di continuità dei coefficienti, \(y(x) \in C^n(I) \implies L(y(x)) \in C^0(I)\);

Inoltre, se \(y_1, y_2 \in C^n(I)\) e \(\lambda_1, \lambda_2 \in \R \), si ha:
\[
    L(\lambda_1 y_1 + \lambda_2 y_2) = \lambda_1 L(y_1) + \lambda_2 L(y_2)
\]

\begin{proof}
    Siano:
    \begin{align*}
        y^{(3)} + xy'' + x^2y = \sin(x) \\
        L(\varphi(x)) = \varphi^{(3)}(x) + x\varphi''(x) + x^2\varphi(x)
    \end{align*}
    \begin{align*}
        L(\alpha\varphi_1(x) + \beta\varphi_2(x)) & = \cdots \\
    \end{align*}
\end{proof}
