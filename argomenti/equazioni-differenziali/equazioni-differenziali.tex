\subsection{Introduzione alle equazioni differenziali}

Le equazioni differenziali sono equazioni in cui l'incognita è la funzione \(y = f(x)\) e i termini sono le derivate della funzione stessa.

Osservazioni:
\begin{itemize}
    \item Equazioni differenziali \(\neq{}\) Equazioni algebriche
    \item Queste equazioni sono utili ad esempio per modellare i fenomeni della realtà in fisica
    \item Servono quando ci serve la funzione incognita \(y(x)\) avendo solo informazioni sul suo comportamento, ovvero le sue derivate che possono ad esempio rappresentare velocità e accelerazione istantanee.
\end{itemize}

\subsubsection{Tipi di equazioni differenziali}

\begin{itemize}
    \item \textbf{EDO}: Equazioni differenziali ordinarie\newline
          funzione incognita di una sola variabile reale. E.g.:\  \(x(t)\)
    \item \textbf{EDP}: Equazioni differenziali alle derivate parziali\newline
          Equazioni con derivate parziali di funzioni di due o più variabili reali
\end{itemize}

Noi vedremo solo alcuni tipi particolari di equazioni differenziali ordinarie (EDO).

\subsubsection{Ordine di un'equazione differenziale}

L'\textbf{ordine} (o grado) è dato dal massimo ordine di derivazione in cui compare la funzione incognita.

Ad esempio, la seguente EDO è di terzo ordine:
\[
    y^{(3)}+2y''+5y = e^x
\]

Noi vedremo quelle del primo e secondo ordine.

\pagebreak
\subsubsection{Forme delle equazioni differenziali}

\defn{Forma implicita}{
Un'equazione di ordine \(n\) è un'equazione del tipo:
\[
    F(x,y(x),y'(x),\ldots,y^{(n-1)}(x),y^{(n)}(x))=0
\]
\[
    F: U \subseteq \R^{n+2} \rightarrow \R
\]
\begin{itemize}
    \item \(y\) è la funzione incognita che dipende da \(x\).
    \item \(F\) è la funzione assegnata che ha valori reali e dipende dalle \(n+2\) variabili: \[x,y(x),y'(x),\ldots,y^{(n)}(x) \]
\end{itemize}
}

\defn{Forma esplicita (o normale)}{
Un'equazione di ordine \(n\) in forma normale è un'equazione del tipo:
\[
    y^{(n)}(x) = F(x,y(x),y'(x),\ldots,y^{(n-1)}(x))
\]
Ovvero se è esplicita rispetto alla derivata di ordine massimo \(y^{(n)}\).
}

\pagebreak
\subsubsection{Soluzioni di un'equazione differenziale}

Risolvere un'equazione differenziale vuol dire trovare tutte le sue soluzioni, o perlomeno studiarne il suo comportamento.

Una soluzione è quindi una funzione \(\varphi=\varphi(t)\) definita su un intervallo \(I \subseteq \R \) e differenziabile \(n\) volte su \(I\) t.c.\  \(\forall x \in I\) è soddisfatta la relazione dell'equazione differenziale.

\defn{Soluzione integrale (curva)}{
    La soluzione di una EDO di ordine \(n\) sull'intervallo \(I\)
    \begin{equation}\label{eq:soluzione}
        F(x,y(x),y'(x),\ldots,y^{(n)}(x)) = 0
    \end{equation}
    \[
        x \in I \subseteq \R
    \]

    è la funzione \(\varphi{}\) definita almeno in \(I\) e derivabile fino all'ordine \(n\) per cui valga (\ref{eq:soluzione}), ovvero:
    \[
        F(x,\varphi(x),\varphi ' (x), \ldots, \varphi^{(n)}(x) ) = 0\ \ \ \forall x \in I
    \]
}

\defn{Integrale Generale}{
L'insieme di tutte le soluzioni di (\ref{eq:soluzione}) in \(I\) si chiama integrale \textbf{generale} di (\ref{eq:soluzione}) in \(I\).

Quello che otteniamo è una formula di 1+ parametri che ci permette di identificare tutte le soluzioni. E.g.:
\[
    y'=2x \quad\text{ha la soluzione generale}\quad y=x^{2}+c
\]
}

\defn{Soluzione particolare}{
È una soluzione unica presa dalla famiglia di soluzione della EDO (\ref{eq:soluzione}), in cui il parametro \(c\) assume un valore costante specifico. E.g.:

\[
    y'=2x \quad\text{ha soluzione particolare}\quad y=x^{2}+1
\]

Il grafico di ogni soluzione è detto \textbf{curva integrale}.
}
