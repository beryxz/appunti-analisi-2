\subsection{Equazioni differenziali del I ordine}

\subsubsection{Forma e soluzioni}

Le EDO di I ordine lineari sono nella forma:

\[
    F(x,y(x),y'(x))=0
\]

\begin{itemize}
    \item \(x\) incognita
    \item \(y = f(x)\) funzione incognita
    \item \(y'=f'(x)\) derivata prima \(\rightarrow \) ciò che sappiamo
\end{itemize}

L'obiettivo è trovare una funzione \(f(x)\) che derivata eguagli la sua derivata prima \(f'(x)\), data solo la sua derivata prima.

\defn{Forma normale}{
    Una EDO nella seguente forma, si dice in forma normale:
    \[
        y'(x) = f(x,y(x))
    \]
}

\defn{Integrale generale}{
    L'insieme di tutte le soluzioni dell'equazione su un intervallo \([a,b]\) dipende da \(x\) e \(c\) e si scrive:
    \[
        \varphi(x,c) = \int f(x) \diff x = y(x) + c \qquad c \in \R
    \]

    Questo è l'integrale generale su \([a,b]\) della EDO di I ordine; rappresenta una famiglia di funzioni che dipendono dalla variabile della funzione e da un parametro.
    \[
        x \mapsto \varphi(x,c)
    \]
    dove il parametro \(c \in \R \) indica tutte le primitive possibili
}

\defn{Soluzione particolare}{
    Se alla EDO imponiamo una condizione supplementare detta \underline{condizione iniziale}, allora, tra tutte le soluzioni ne selezioniamo una singola chiamata \textbf{soluzione particolare}.
    Questa soluzione particolare è l'unica che verifica la condizione iniziale.

    Un esempio di condizione iniziale:
    \[
        y(x_0) = y_0 \in \R
    \]
}
