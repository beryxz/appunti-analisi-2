\subsection{Esame 2024{-}01{-}12}

\begin{enumerate}
    \itemsep32pt
    \item Risolvere il seguente problema di Cauchy:

          \begin{equation*}
              \begin{cases}
                  y'(x) = -\frac{\sin{2x}}{1+\cos^{2}{x}}y(x) + \sin{x} \\
                  y(\pi) = 1
              \end{cases}
          \end{equation*}

          e precisare qual è il più ampio intervallo su cui la soluzione è definita.

    \item Calcolare il seguente integrale doppio:

          \[
              \iint_{D} xe^{x^2}e^{|y+x+1|} \diff{x}\diff{y}
          \]

          dove \(D = \{(x,y) \in \R^{2} \giventhat |x| - 1 \le y \le 1 - x^2,\ x \le 0\} \).

    \item Determinare, se esistono, i punti di minimo e massimo assoluti della funzione

          \[
              f(x,y) = xye^{-xy}
          \]

          sul dominio \(D = \{(x,y) \in \R^{2} \giventhat 0 \le x,\ 1 \le y \le 2,\ |xy| \le 1 \} \).

    \item Data la funzione

          \begin{equation*}
              f(x,y) =
              \begin{cases}
                  \frac{x^2{(y-1)}^3}{{\left[x^2+{(y-1)}^2\right]}^{\frac{5}{2}}} & (x,y) \neq (0,1) \\[4mm]
                  3                                                               & (x,y) = (0,1)
              \end{cases}
          \end{equation*}

          stabilire se è continua sul suo dominio di definizione.

\end{enumerate}
