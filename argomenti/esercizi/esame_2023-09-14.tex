\subsection{Esame 2023{-}09{-}14}

\begin{enumerate}
    \itemsep32pt
    \item Risolvere il seguente problema di Cauchy:

          \begin{equation*}
              \begin{cases}
                  y^2y' = 2x +1 \\
                  y(0) = -1
              \end{cases}
          \end{equation*}

          e precisare qual è il più ampio intervallo su cui la soluzione è definita ed è di classe \(C^1\).

    \item Calcolare il seguente integrale doppio:

          \[
              \iint_{D} \left(\left|\sqrt{3}-y\right|-\sqrt{3}x\right) \diff{x}\diff{y}
          \]

          dove \(D = \left\{ (x,y) \in \R^{2} \giventhat x^2+y^2 \le 1,\ y \ge \frac{x}{2} - \frac{|x|}{2} \right\} \)

    \item Determinare i punti critici (stazionari) della seguente funzione e studiarne la natura (stabilire se sono punti di minimo, massimo o sella):

          \[
              f(x,y) = (x^2+y^2)(1-y)
          \]

    \item Calcolare, se esiste, il seguente limite:

          \[
              \lim_{(x,y) \rightarrow (0,0)}
              \frac
              {(e^x-e^{2y})(x-y)}
              {x^2 + y^2}
          \]

\end{enumerate}
